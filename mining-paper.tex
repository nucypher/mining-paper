\documentclass[longbibliography,nofootinbib,twocolumn]{revtex4-1}

\newcommand{\kms}{NuCypher KMS}

\usepackage{listings}
\usepackage{graphicx}
\usepackage{amsmath}
\usepackage[margin=5pt]{subfig}
\usepackage[usenames]{color}

\renewcommand{\baselinestretch}{1.4}
\setlength{\parskip}{1em}
\definecolor{darkgreen}{rgb}{0.00,0.50,0.25}
\definecolor{darkblue}{rgb}{0.00,0.00,0.67}
\newcommand{\figref}[1]{Fig.~\ref{#1}}
\usepackage[breaklinks,pdftitle={NuCypher KMS: Mining}, pdfauthor={Michael Egorov},colorlinks,urlcolor=blue,citecolor=darkgreen,linkcolor=darkblue]{hyperref}
\graphicspath{{pdf/}}

\usepackage[T1]{fontenc}
\usepackage{lmodern}
\lstset{
    basicstyle=\ttfamily,
    basewidth={0.5em, 0.5em},
    columns=fullflexible,
}

\begin{document}

\title{\kms: Mining}

\author{Michael Egorov}
\email{michael@nucypher.com}
\affiliation{NuCypher}

\begin{abstract}
    This paper describes mining mechanisms and economics in \kms.
    It includes inflation rates, mechanisms to incentivise long-term stakers
    and estimates of number of coins generated by nodes running in typical modes.
    Also, optimal strategies for stakers who may be affected by market volatility are proposed.
\end{abstract}

\date{\today}
\maketitle

\section{Motivation}

In future, \kms~will probably be fully paid by network fees.
But initially, when the adoption isn't yet high, miners who run the nodes necessary for network operation and keep re-encryption keys,
will need to be subsidised.
This will be done through inflation schedule, where all the inflation is given back to miners.

Distribution of rewards should have the following properties:
\begin{itemize}
    \item All the inflation is distributed to stakers who run the nodes, proportionally to the stake;
    \item Amount of work (and, hence, the fees) is proportional to stake also;
    \item Stakers are incentivized (by a higher reward rate) to run long-term nodes;
    \item High inflation doesn't depreciate the price in order to keep liquidity good for new stakers;
    \item Stakers are incentivized to stay online all the time.
\end{itemize}

In the paper we address all these points, calculate expected earnings of miners who run nodes and devise optimal mining strategies.

\section{Historical examples of inflation}

Let's review inlation schedules of different cryptocurrency projects:
DASH~\cite{dash:whitepaper}, ZCash~\cite{zcash} and Steam.it~\cite{steamit}.

\bibliography{mining-paper}

\end{document}
