\documentclass[longbibliography,nofootinbib,twocolumn]{revtex4-1}

\newcommand{\kms}{NuCypher KMS}

\usepackage{listings}
\usepackage{graphicx}
\usepackage{amsmath}
\usepackage[margin=5pt]{subfig}
\usepackage[usenames]{color}

\renewcommand{\baselinestretch}{1.4}
\setlength{\parskip}{1em}
\definecolor{darkgreen}{rgb}{0.00,0.50,0.25}
\definecolor{darkblue}{rgb}{0.00,0.00,0.67}
\newcommand{\figref}[1]{Fig.~\ref{#1}}
\usepackage[breaklinks,pdftitle={NuCypher KMS: Mining}, pdfauthor={Michael Egorov},colorlinks,urlcolor=blue,citecolor=darkgreen,linkcolor=darkblue]{hyperref}
\graphicspath{{pdf/}}

\usepackage[T1]{fontenc}
\usepackage{lmodern}
\lstset{
    basicstyle=\ttfamily,
    basewidth={0.5em, 0.5em},
    columns=fullflexible,
}

\begin{document}

\title{\kms: Mining}

\author{Michael Egorov}
\email{michael@nucypher.com}
\affiliation{NuCypher}

\begin{abstract}
    This paper describes mining mechanisms and economics in \kms.
    It includes inflation rates, mechanisms to incentivise long-term stakers
    and estimates of number of coins generated by nodes running in typical modes.
    Also, optimal strategies for stakers who may be affected by market volatility are proposed.
\end{abstract}

\date{\today}
\maketitle

\section{Motivation}

In future, \kms~will probably be fully paid by network fees.
But initially, when the adoption isn't yet high, miners who run the nodes necessary for network operation and keep re-encryption keys,
will need to be subsidised.
This will be done through inflation schedule, where all the inflation is given back to miners.

The individual inflation rate will be dependent on the minimal time $\tau$ the node commits to mine for.
It will incentivize longer miners.
The coins staked at this time will be fully available in no less than the time $\tau$.

Mining rates, however, shouldn't be overly too high: too high inflation schedules may depreciate the price.
For example, the price of ZCash experienced an inflection point (started going up) only when it become lower than $350\%$ APR.
However, for steam.it, the inflation rate of about $100\%$ per year appeared to be too high.
Hence, to be on the safe side, our inflation should be lower than $100\%$ per year, and also we should provide convenient ways for miners to restake what they
mined.
If the restaking happens automatically, so that what is mined isn't even taken out of the smart contract, it could have tax advantages in some jurisdictions
also.

\section{Incentives to create long-term stakers}

\section{Inflation models}

\section{Possible strategies for stakers}

\section{Edge cases: restaking during unlocking, connectivity problems}

\end{document}
